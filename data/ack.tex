%%================================================
%% Filename: ack.tex
%% Encoding: UTF-8
%% Author: Yuan Xiaoshuai - yxshuai@gmail.com
%% Created: 2012-01-12 18:09
%% Last modified: 2016-08-28 21:05
%%================================================
\begin{ack}
	在这个充满感慨的毕业季,我用一篇毕业论文来为我的大学四年画上一个温柔的句点。回顾这段旅程,我深刻地感受到,计算机的知识犹如建造房屋时的基石,只有扎实的知识基础才能打造出坚固的地基;而一篇论文的框架和清晰的逻辑思维,则是那座房子坚实的骨架和美丽的外观。这次的毕业论文之旅,更是让我领悟到,独立思考和善于发现问题,是作为一名未来的计算机科学家所应具备的宝贵品质。

	在此,我首先要深深感谢我的指导老师,张格老师。从论文的选题到最终定稿,张格老师用他那最负责任和认真的态度,为我们指点迷津。在这条学术之路上,您像一位严格的守门人,对我们的论文进行精准的把控,为我们的疑惑提供专业的指导和建设性的建议,使我们在论文的修改过程中如获至宝。对张格老师的辛勤付出,我怀着最诚挚的感激之情。

	我还要感谢那些在我写作过程中伸出援手的同学们。在我迷茫和困惑时,是你们的帮助让我看到了希望,使我的毕业论文得以顺利完成。我们共同度过的日夜,将成为我最宝贵的记忆。

	不得不提的是科技的力量,如果没有日新月异的技术发展,许多相关的实验就不可能实现。我还要感谢那些在网络上给予我指导的网友,没有你们的指点,我的论文就会失去方向。同时,对那些论文参考文献的作者们表示感谢,你们的理论支持和实验数据构成了我论文的骨干。

	在此,我还要对我的父母表达最深的感激。是你们在背后默默地支持我,无论是物质上还是精神上,都给予了我最坚实的后盾,让我能够在一个安稳的环境中专心完成我的写作。

	大学四年的时光如同一部精彩的小说,即将落下帷幕。我要感谢所有任课老师四年来的悉心教导。对于老师们指出的问题,我会虚心接受,并积极改进。在此,我再次向所有帮助过我的老师和同学们表达我最真挚的感谢。这段旅程因你们而精彩,这份成长因你们而丰富。

	\begin{center}
		请君试问东流水,别意与之谁短长。

		山水有相逢,来日皆可期。
	\end{center}

\end{ack}
