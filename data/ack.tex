%%================================================
%% Filename: ack.tex
%% Encoding: UTF-8
%% Author: Yuan Xiaoshuai - yxshuai@gmail.com
%% Created: 2012-01-12 18:09
%% Last modified: 2016-08-28 21:05
%%================================================
\begin{ack}
在毕业之际,以毕业论文的方式结束大学四年的美好时光。在撰写毕业论文的过程中,深刻体会到法学知识就像建房子,扎实的知识基础才会有牢固的地基,构建好论文的框架和清析的逻辑思维才能写好一篇有质量的论文。此次的毕业论文也让我知道,独立的思考善于发现问题是作为一名计算机人应具备的美好品质。

本次毕业论文能顺利完成首先最感谢的是我的指导老师张格老师,从我们论文的选题、开题报告、一稿、二稿、定稿和最终稿,张格老师都以最负责、最认真的态度给我们进行指导,对学生的毕业论文进行严格把关,针对我们的论文进行专业的指导,针对我们存在的问题,给予有建设性的建议,给学生进行论文修改带来了很多便利。对张格老师表示诚挚的感谢。其次也非常感谢对我予以帮助的同学,感谢你们在我写作的过程中,遇到困惑时给予我帮助,对于我的毕业论文完成也起到关键作用。

感谢科技的发展,如果没有日新月异的大模型,就不能做出很多相关的实验。感谢网友的网友的指导,没有你们就没有论文的方向,感谢论文的参考文献的作者们,你们提供给我的理论支持和实验数据,是我论文的骨干。感谢父母背后对我的资金的支持和精神上的支持,没有你就没有安稳地写作环境。

大学四年即将画上圆满的句号,在此也感谢各位任课老师大学四年的教导。对于论文老师指出存在的问题,我会虚心听取,积极的改进。在此再次向各位帮助过我的老师和同学们表示诚挚的感谢。

\begin{center}
    请君试问东流水,别意与之谁短长。

    山水有相逢,来日皆可期。
\end{center}

\end{ack}
