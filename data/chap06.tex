%%================================================
%% Filename: chap06.tex
%% Encoding: UTF-8
%% Author: 苏峻锋
%%================================================
\chapter{总结与展望}

\section{本文工作总结}
现在互联网发展愈来愈迅速,服务器集权架构越来越复杂,用户对各种服务的要求也越来越高,不仅各种中小型企业甚至大型企业也很重视对于复杂均衡的研究,以达到更好的性价比。本文在背景下,对现在常用的负载均衡策略和基本的时间序列预测模型进行了较为充分的理解和研究,针对了具体的访问量预测方面和Nginx内置的平滑加权轮询算法进行了与神经网络的深度结合,很好的处理了节点调整权重的问题。通过对于时间序列分析和预测以及对传统的负载均衡算法策略的深入研究,提出了基于时间序列预测模型的动态加权轮询算法;对如何进行合理分配的负载分配和权重的调整进行了深入研究,对加权轮询算法进行的改进,从而使得较好的利用好服务器集群的剩余性能,从而达到降低花销,提高服务器整体负载性能。本文完成的工作内容有以下几个方面。

(1)研究常规负载均衡算法涉及的相关理论和技术,分析了负载均衡的技术意义和优势,了解了集群的概念和分类,通过对Nginx源码的了解探究了Nginx具体的工作模式和进程模型,确认了整体的研究方向。

(2)根据 Kaggle 比赛 “Web Traffic Time Series Forcasting“所提供的网络访问量数据集,运用 ARIMA 模型和 prophet 模型对网络访问量进行预测,并针对不同情况的访问量对比两个模型的性能。最终确定了 prophet 在预测网络访问量的性能优势和准确性。

(3)针对 Nginx 内部自带的加权轮询算法不能实时监控各个服务器节点的负载信息的问题,对峙服务器集群不能合理地对请求任务进行分配,制定了在各个服务器节点周期性进行收集并存入 Memcached,尽量减少对于节点资源的消耗。同时创新的设置了两个不同的阶段,解决了改进算法初期没有数据,初期集群不稳定的难题。

本文的创新点如下:

对于传统负载均衡没有考虑到的可能访问量的问题提出了使用 prophet 模型来进行访问量预测,并证明了 prophet 模型具有较高的准确度和性能。针对负载均衡算法中动态调整权重的问题,对加权轮询算法进行了一定的改进,引入拥塞控制思想,并分为两个不同的阶段,解决了初期没有数据的问题,根据节点负载预测结果对服务器节点进行调整,提高了集群系统运行期间权重分配的合理性,减少了负载调整的滞后性。

\section{工作展望}

本文对 Nginx 在负载均衡问题上做了许多研究和分析,与自带的负载均衡策略相比,本文提出的优化后的预测模型动态加权轮询算法一定程度上解决了传统算法的弊病,但是仍旧存在很多的不足。

(1)由于实验条件和笔者技术条件的限制,此次研究没有合适的具体实验数据来对改进的算法进行实验,但是证明了 prophet 模型在访问量预测的有效性,若改进的算法在复杂的服务器集群中,有可能会产生一系列其他的问题。

(2)本文中用户的访问请求全部交给了负载均衡服务器,通过它使用的负载均衡算法进行合理的任务分发。但是如果当负责负载均衡的服务器宕机或者资源利用达到上线,那么会造成整个集群无法正常工作甚至瘫痪,为了解决该问题可以采用 Nginx 和 KeepAlive 双热机主从备份的方法,保证服务器集群的正常运行。
