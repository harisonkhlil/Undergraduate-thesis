%%================================================
%% Filename: chap06.tex
%% Encoding: UTF-8
%% Author: 苏峻锋
%%================================================
\chapter{总结与展望}

\section{本文工作总结}

随着互联网技术的持续快速进步,服务器的集中管理架构变得日益复杂,对于服务的需求标准也在不断提升。
这一趋势不限于中小企业,甚至在大型企业中也显得尤为明显,它们高度重视对于负载平衡技术的研究,旨在通过优化成本效益比实现更高的性能表现。
在此背景下,本文对当前流行的负载平衡策略及基础时间序列预测模型进行了全面的探究和分析,
尤其是将重点放在了访问量预测和Nginx的平滑加权轮询算法与神经网络的结合应用上,理论上有效解决了在节点权重调整过程中存在的问题。
通过深入的时间序列分析和预测,以及对传统负载平衡算法的深度研究,
本文提出了一种基于时间序列预测模型的动态加权轮询算法。此外,本文也对负载的合理分配与权重调整方法进行了详尽的探讨,
并对加权轮询算法进行了优化。这些改进有助于更有效地利用服务器集群的剩余资源,从而有效的降低企业的成本问题,以及服务器的负载问题。
本文的主要贡献可以概括为以下几点。

(1)研究常规负载均衡算法涉及的相关理论和技术,分析了负载均衡的技术意义和优势,了解了集群的概念和分类,通过对Nginx源码的了解探究了Nginx具体的工作模式和进程模型,确认了Nginx 负载均衡算法改进优化的整体研究方向。

(2)根据 Kaggle 比赛 “Web Traffic Time Series Forcasting“所提供的网络访问量数据集,运用 ARIMA 模型和 prophet 模型对网络访问量进行预测,并针对不同数据的访问量对比两个模型的预测准确度。最终确定了 prophet 在预测网络访问量的性能优势和准确性。

(3)针对 Nginx 内部自带的加权轮询算法不能实时监控各个服务器节点的负载信息的问题,服务器集群不能合理地对请求任务进行分配,制定了在各个服务器节点周期性进行收集并存入 Memcached,尽量减少对于节点资源的消耗。同时创新的设置了两个不同的阶段,解决了改进算法初期没有数据,初期集群不稳定的难题。

本文的创新点如下:

对于传统负载均衡没有考虑到的可能访问量的问题提出了使用 prophet 模型来进行访问量预测,以此进行服务器性能的储备,并证明了 prophet 模型具有较高的准确度和性能。
对负载均衡算法中动态调整不同任务下不同服务器权重的问题,对平滑加权轮询算法进行了一定的改进,加入了周期性模块,考虑了两个不同的阶段,
根据节点负载预测结果对服务器节点进行调整,提高了集群系统运行期间权重分配的合理性,减少了负载调整的滞后性。

\section{工作展望}

本文基于对 Nginx 负载均衡问题上做了许多研究和分析,本人改进的负载均衡算法与传统的负载均衡策略相比,
本文提出的优化后的预测模型动态加权轮询算法一定程度上解决了过去算法的弊病,但是仍旧存在很多的缺点。

(1)由于实验条件和笔者技术条件的限制,此次研究没有合适的具体实验数据来对改进的算法进行实验,但是证明了 prophet 模型在访问量预测的有效性,分析给出了综合负载指标的时间序列预测具体方法。但是,如若改进的算法在复杂的业务和具体情境中中,有可能会产生一系列其他的问题。

(2)本文中几乎所有的任务都需要通过负载均衡器来进行控制和调节,通过负载均衡器来接受用户请求,通过它使用的改进后的负载均衡算法进行不同权重的的任务分发。
但是如果当作为负载均衡器的服务器宕机或者资源利用达到上限,那么会造成整个服务器集群无法正常工作甚至瘫痪。
为了解决该问题,可以采用 Nginx 和 KeepAlive 双热机主从备份的方法,从而保证服务器集群的正常运行。

(3)负载均衡算法使用期间有着“既当运动员又当裁判员”的嫌疑,单单考虑了集群内的各种负载指标,通过负载阈值同时更新负载参数,有时并不能有很好的效果,这就需要结合集群以外的指标,例如最小响应时间等。
