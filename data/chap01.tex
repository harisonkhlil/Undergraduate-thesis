%%================================================
%% Filename: chap01.tex
%% Encoding: UTF-8
%% Author: Yuan Xiaoshuai - yxshuai@gmail.com
%% Created: 2012-04-27 15:05
%% Last modified: 2016-08-28 21:07
%%================================================
\chapter{\TeX/\LaTeX{}系统概述}
\label{cha:overview}

\TeX{}\footnote{\TeX{}的名称是由三个大写的希腊字母ΤЄΧ组成,在希腊语中这个词是“
科学”和“艺术”的意思。为了方便的缘故,一般都写成“TeX”,念做“teck”。}是一个格式
化排版系统,它一问世便以其排版效果的高质量震动整个出版界。尤其是在排版含有大量
数学公式的科技文献方面更显示了它的优越性。\TeX{}还是一个程序源代码公开的免费排
版系统,因此吸引了许多计算机专家及\TeX{}爱好者为之添砖加瓦。

\TeX{}不仅是一个排版程序,而且是一种程序语言。\LaTeX{}\footnote{\LaTeX{}的读法
应为“lay--teck”,念成“lay--tecks”也可以。}就是使用这种语言写成的一个“\TeX{}宏
包”,它扩展了\TeX{}的功能,使我们很方便地进行富于逻辑性的创作而不是专心于字体
、缩进等这些烦人的东西。

\TeX{}是\LaTeX{}的基石,\LaTeX{}建立在\TeX{}之上;各种宏包和类型文件是\LaTeX{}
大厦的装饰材料。\LaTeX{}是特殊版本的\TeX{},并且加进了很多新功能,使得使用者可
以更为方便的利用\TeX{}的强大功能。使用\LaTeX{}基本上不需要使用者自己设计命令和
宏等,因为\LaTeX{}已经替你做好了。因此,即使使用者并不是很了解\TeX{},也可以在
很短的时间内制成高质量的文件。对于排版复杂的数学公式,\LaTeX{}表现的更为出色。

\LaTeX{}与Word是两种不同类型的文本编辑处理系统,各有所长,如果要对文字编辑性能
和使用便捷程度等作综合评比,Word明显优于\LaTeX{},仅“所见即所得”一项,Word就会
赢得绝大多数用户,但要仅限定在学术报告和科技论文方面,评比结果就不同了:

\section*{从头开始}

Word特点就是“所见即所得”,其基本功能初学者很容易掌握,很多Word用户都是无师自通
。但随着篇幅和复杂程度的增加,花费在文稿格式上的精力和时间要明显加大。因为创建
自定义编号、交叉引用、索引和参考文献等就不是“所见即所得”了,得耐着性子反复查阅
Word的在线帮助或借助相关软件帮忙。

对于\LaTeX{}初学者,即就是编排很简单的文章,也要花较多的精力和时间去学习那些枯
燥的命令和语法,特别是排写数学公式,经常出错,多次编译不能通过,使很多初学者望
而却步。可是一旦掌握,不论文稿长短和复杂与否都会熟练迅速地完成,先前学习
\LaTeX{}的精力投入将由此得到回报。

\section*{内容与样式}

当用Word写作时,要花很多精力对页版式、章节样式、字体属性、对齐和行距等文本参数
进行反复选择对比,尤其是长篇文章,经常出现因疏忽而前后文体格式不一致的现象;当
在稿件中插入或删除一章或章节次序调整时,各章节标题、图表和公式等的编号都要用手
工作相应修改,稍有不慎就会出现重号或跳号。你既是作者又是编辑还兼排字工。

使用\LaTeX{}编版,如无特殊要求,只要将文稿的类型(article、report或book等)告
诉\LaTeX{},就可专心致志地写文章了,至于文稿样式的各种细节都由\LaTeX{}统一规划
设置,而且非常周到细致;当修改稿件时,其中的章节、图表和公式等的位置都可任意调
整,无须考虑编号,因为在源文件里就没有编号,文件中的所有编号都是在最后编译时
\LaTeX{}自动统一添加的,所以绝对不会出错。

换句话说,Word把文稿的内容与样式混为一体,而\LaTeX{}将它们分离,作者只需专注于
文稿的内容,而文稿的样式几乎不用过问,\LaTeX{}是你的聪明而忠诚的文字秘书,如有
特殊要求,也可使用命令修改,\LaTeX{}会自动将相关设置更新,无一遗漏。

接受\LaTeX{}稿件的出版社大都有自己的文稿样式模板,主要就是一个类型文件包,简称
类包。如果稿件未被甲出版社采用,在转投乙出版社前,只需将稿件第一句中类包名称由
甲出版社的改为乙出版社的,整篇稿件的样式就随之自动转换过来了。就一句话的事儿,
简单的不能再简单了,然而因为“体制”的原故,Word却根本无法做到这一点。

\section*{数学公式}

Word有个公式编辑器,可以编辑普通数学公式,但使用很不方便,外观效果较差,也不能
自动编号,尤其是很难作为文本的一部分,融入某一行中,大都专起一行。如果碰到复杂
的数学公式,编辑起来就很困难。有些用户只好另外安装可嵌入Word环境的工具软件
Math-Type来弥补这一不足。

\LaTeX{}的特长之一就是数学公式编辑,方法简单直观,“所想即所得”,公式的外观精致
细腻,而且公式越复杂这一优点就越明显。普通单行公式可以像纯文字文本一样插入字里
行间。尽管在默认状态下,就能将数学公式编排的非常精致美观,\LaTeX{}仍然还提供了
很多调节命令,可以对公式的外观作更加细微的调整,使其尽善尽美。

\section*{插图}

Word有个绘图工具,简易直观,但功能有限效果不佳。论文中的复杂图形大都用功能强大
的Visio、Photoshop等绘图软件绘制,然后插入Word。

\LaTeX{}自身也具有简单的绘图功能,如调用各种绘图宏包,可画出非常复杂的图形,缺
点是不直观,命令格式繁琐,不易熟练掌握,名曰画图,实为编程。可同样先使用Visio
绘图,然后粘贴到Adobe Illustrator,对图形的细节作进一步处理后,存储为PDF或EPS
格式,最后用插图命令调入\LaTeX{}源文件即可,其效果更为精致。

\section*{创建参考文献}

Word目前还不具备管理参考文献的功能,用户一般都是采用Reference Manager或是
NoteExpress等外部工具软件来解决这一问题。

创建参考文献可是\LaTeX{}的强项。\LaTeX{}自带一个辅助程序\BibTeX{},它可以根据
作者的检索要求,搜索一个或多个文献数据库,然后自动为文稿创建所需的参考文献条目
列表。如果编写其它文件用到相同的参考文献时可直接引用这个数据库。参考文献的样式
和排序方式都可以自行设定。

很多著名的科技刊物出版社、学术组织和TUG网站等都提供相关的\BibTeX{}文献数据库文
件,可免费下载。

\section*{显示与输出}

在文本对齐、字体变换、拼写检查、单词间距控制、自动断词和自动换行等纯文字处理功
能方面,Word经多次升级后已与\LaTeX{}相差无几,但是排版效果却有所不同。以Times
字体为例,在Word中“Ta”和“PA”两个字母的间距有些松散。\LaTeX{}将各种拼写组合时的
字母间距进一步优化调整,松紧得当,使整个文本的排版效果更加工整匀称。

在换行时,\LaTeX{}不仅可以根据音节自动断词,也可以按照作者的要求进行设定断词,
一个单词可以设定多种断词方式,特别适用于科技论文中反复出现的专业词汇或缩略写,
这既能保持单词间距均匀,又不易产生误解。

在科技著作手稿中经常可以看到某些论述附有说明、出处或考证;或者某些段落划上黑杠
以示删除;或在边空里写有准备补充的文字。在\LaTeX{}源文件中使用注释标记可以将上
述这些内容完整地保留下来,以备后用,而在编译后的PDF文件中还看不到这些内容。科
研论文要经过反复推敲,多次修改,注释功能非常实用。“所见即所得”的Word,当然没有
这个功能,它删除的内容就甭想再找回来了。

一篇论文,Word新手与牛人的排版美观程度差别很大,“所见即所得”成了一大缺点,因为
Word本身不能帮助作者美化作品,自己排成什么样就什么样,即:“所得仅所见”,就像在
白纸上作画,全凭个人的悟性与灵感。而\LaTeX{}初学与专家的排版外观差别很小,仅是
快慢不同,都能达到专业出版水平,这就是\LaTeX{}的一大优点,只要想法一致就能得到
相同的结果,也就是“所想即所得”。

目前PDF格式已成为全世界各种组织机构用来进行更加安全可靠的电子文件分发和交换的
出版规范,科技论文大都使用PDF格式。\LaTeX{}可以直接输出PDF、PS或DVI格式文件;
而Word输出的是DOC格式文件,还须将DOC转换为PDF;另外,图形中的数学公式或文本中
数学式的上下标,在转换后常出现位置偏移字形变大等问题。

\section*{可扩充性}

用户可以像搭积木那样对\LaTeX{}进行功能扩充或添加新的功能。例如,加载一个CJK宏
包,就可以处理中文,调用eucal宏包可将数学公式中的字符改为欧拉书写体;如果对某
个宏包效果不太满意,完全可以打开来修改,甚至照葫芦画瓢自己写一个。这些可附加的
宏包文件绝大多数都可从CTAN等网站无偿下载。

因为设计的超前性,\TeX{}/\LaTeX{}程序系统几十年来没有什么改动,而且由于它的可
扩充性,\LaTeX{}将永葆其先进性,也就是说,学习和使用\LaTeX{}永远不会过时。例如
,通过调用相关扩展宏包,\LaTeX{}立刻就具备了排版高质量高专业水准象棋谱、五线谱
或化学分子式的能力。对于\LaTeX{}这种机动灵活、简便免费的可扩充性能,Word只能望
尘。

\section*{稳定性和安全性}

一篇科技论文少则几十页,多则上百页,其中含有许多图形和公式(Word将公式处理为图
形),正是由于Word“所见即所得”,论文中的图形都要完整地插入页面。随着文件的篇幅
增大图形数量增多,处理速度明显减慢。编写一篇论文要无数次地打开、保存和关闭,往
往要长时间等待甚至死机或文稿无法打开,所以Word经常出现“文件恢复”提示信息,但其
中的图形很有可能丢失,取而代之的是一个小红叉。如果将文件分解为多个子文件,可以
缓解这一问题,但又会出现难以自动创建目录、索引和参考文献等新问题;若章节、图表
和公式需要在子文件之间调换调整,那编号就全乱套了。

\LaTeX{}是纯文本文件,所有图形都是在最后编译时调入。同一篇文章,其\LaTeX{}源文
件只有Word文件尺寸的几十分之一。所以,\LaTeX{}源文件的长短,不会对文件存取和编
辑过程产生明显影响。\LaTeX{}也允许采用多个子文件,章节和图表可随意增删,
\LaTeX{}是在最后编译时才将所有子文件汇总排序,生成统一的文件页码、标题序号、图
表和公式编号以及各种目录。

Word从问世到现在不断地更新版本,并经常要求下载补丁程序,防止病毒攻击。\LaTeX{}
及其前身\TeX{},近二十年来,没有发现系统漏洞,即使有,造成源文件损坏的风险也是
微乎其微;迄今也未发现任何宏包含有病毒。

\section*{版本兼容性}

Word十几年里已有多种版本,只能向下兼容,旧文件在新版本中打开,经常出现字形和文
本位置变动等问题。

二十年来\LaTeX{}也有几种版本,但可相互兼容,旧文件在新版本中打开,文本不会有丝
毫的变形,而且还可以继续追加新的功能,如这几年很流行的超文本链接和PDF书签等。

\section*{通用性}

随着计算机软硬件性能的提高,在PC机上使用Unix/Linux、MacOS或其他操作系统的用户
越来越多。由于\LaTeX{}系统的程序源代码是公开的,因此人们开发了用于各种操作系统
的版本,而且\LaTeX{}源文件全部采用国际通行的ASCII字符,所以\LaTeX{}及其源文件
可以毫无阻碍地跨平台、跨系统使用和传播。

而Word只能在Windows操作系统上运行。

高德纳教授曾说过:TeX排版系统追求的首要目标就是高品质,文件的排印效果不只是很
好,而是要最好。\LaTeX{}就是专门为排版高质量科技论文而设计的软件,当然在这方面
的性能就非常突出。在很多\LaTeX{}爱好者看来,\LaTeX{}不仅是一种文字编辑排版工具
,它更是一门艺术,给人以美的享受。然而,追求完美是要付出一定代价的,是否值得,
那得您说了算!

