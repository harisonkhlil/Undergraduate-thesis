%%================================================
%% Filename: chap01.tex
%% Encoding: UTF-8
%% Author: Su JunFeng
%% Email: harisonkhlil@gmail.com
%% Created: 2024-02-21
%% Last modified: 2024-02-22
%%================================================
\chapter{绪论}

\section{研究背景及意义}

\subsection{研究背景}

随着互联网的高速发展和广泛普及,人们的生活与互联网已密不可分。
从年轻的数字原住网民到如今老年人的网络化,都在积极参与数字世界,展现出强烈的社交化、互动性和创造力,形成了独特的网络文化。
根据中国权威调查中国互联网络信息中心调查研究第52次《中国互联网络发展状况统计报告》显示,
截止2023年6月,我国网民规模达10.79亿次,较2021年12月增长3549万,互联网普及率达75.6\%\cite{vsgohulmwhlofavjvlkltsjibcgc}

网络范围的不断扩大,服务人民的服务器日益增加,利用互联网收发消息,进行娱乐视频的行为越来越多。
在这种境地下,对后台服务器的处理能力提出了严峻的挑战。前人发明了许多方法来应对这种挑战。
疏导,引流,集群,但是随着集群规模的增加,也就是服务器设备的不断增加。服务器的不同应用还会频繁产生互动行为,
在这种互动行为中,集群中服务器的交流,远程调用是需要解决的第一个难题,而且当用户提交的请求不断增加时,怎么样保持节点的负载状态稳定?
当用户发出请求任务,为了保证节点的的负载状态稳定,多个服务器就会采取一种策略对任务进行分配,以合理的调取资源,这就是负载均衡算法。

由于网络请求量不停地增长,要提高服务器处理请求的性能可以从两个方面提升。
一方面是硬件,也被称为硬件负载均衡。增加后端服务器节点的数量来达到分摊负载的目的,提高集群整体的负载能力。
但是该方法需要的成本巨大,除了一些大型企业有能力购买之外,对于其他中小型企业来说负担太重。
而且即便花费了企业资金买进了很多高性能的服务器,可如果无法使这些设备完全展现出它们的能力,
也会使得性价比降低。但是显然一些中小型企业是不会这样做的,他们选择从另一方面进行优化,即软件算法上的优化,
这样既节省资源还可以提高服务器性能。\cite{qbee}

综上所述,除了大型企业,其它类型的企业从软件方面进行改进是一个可行的方案,
而且对于大型企业来说进行软件方面的改进可以使庞大的服务器集群的性能更上一层楼。
软件方面进行的负载均衡被称为软负载均衡,在一台或者多台服务器上不同的操作系统中安装一个
或者多个附加软件以实现负载均衡。这种方法最大的特点就是配置可能更加简单,价格比较亲民。

对于软负载均衡常见的有 LVS\cite{lijp}, Nginx\cite{Zepeng}, HAProxy\cite{li2019dynamic} , Ribbon 等等。其中,Nginx 是一款优秀的软件负载均衡器,具有并发量高代码开源等优点,因此常常被用来作为服务器端的负载均衡器。
本文也是基于 Nginx 负载均衡系统的研究,在研究的基础上探究负载均衡算法并创造优化算法

\subsection{国内外研究现状}

几十年来,互联网发展突飞猛进,Web 服务器所需要完成的并发请求量日益增加,带来的
就是服务器处理压力不断增加,在除了硬负载均衡的进步以外,在软负载均衡研究中国内外
也提出了并实现了一系列的优化,出现了一系列服务器软件,最为著名的就是 Nginx。

在国外,XIAONICHI等人\cite{chi2012web},分析研究了负载均衡技术,
参考前人运用这些技术手段做过的一些成功案例,并且重点研究了Nginx的主要功能模块和内核,
运用Nginx作为反向代理服务器来解决并发难题。运用负载均衡技术和缓存技术来应对出现服务器过载的情况。
并且结合实际情况来设计网络拓扑架构,将 Discuz 论坛应用到该网络架构中,
实验仿真分析后得出的结果,证明了基于Web缓存技术可以很好的解决高并发的问题。

同时,国外的一些企业也一直研究用户集群系统的负载调度器。
Cisco的LocalDirect推出最快响应算法和最小连接数算法。
IBM的NetworkDispatcher选择当前连接数最少的服务器处理新的请求任务。
Intel公司网擎的负载调度策略是采用快速响应算法\cite{张淇2020服务器集群负载均衡算法在商务系统中的研究与应用}。

在国内,2016年,王永辉采用集群服务器分组管理算法,在各服务器周期性采集完负载指标信息后向集群控制节点上报,
由控制节点向负载均衡器统一反馈,负载均衡器针对负载信息计算每个服务器的加权值,
并选出权值最大的三台服务器,以正比于权值的概率随机选择一台服务器处理客户端请求\cite{王永辉2015基于}。

2018年,王东提出一种基于UDP协议多播实现服务器负载指标数据传输的动态负载均衡算法,
该算法将一段时间内多核CPU的平均工作负载作为阈值对多核CPU是否达到最大工作负载进行判断,
由此负载均衡器将减少CPU处于满载的服务器的任务量\cite{王东2018动态反馈负载均衡策略的研究}。
实验证明该算法有效提升了系统的吞吐率,但UDP传输方式仍然存在一定的可靠性问题。

2021 年谭畅,胡磊等人\cite{谭畅2021云中心基于}提出了一种动态调节权重的负载均衡策略。
该策略基于加权轮询策略进行改进,同时考虑服务器的本身硬件性能与工作时的负载情况,
经过实验测试结果表明相较于原加权轮询算法,该算法在高并发情况下的响应时间
和实际并发数等方面表现更好。

\subsection{研究目的与意义}

在阅读他人或企业组织关于负载均衡的论文时,发现了许多有关于使用机器学习,深度学习,神经网络等方法实现
负载均衡算法的预测从而处理高并发,降低负载率。

2012 年通过数学多元回归分析方法计算得到系统的响应时间和反映节点负载状态两个参数之间存在
的联系,并通过这个联系建立关系模型进行服务器负载的预测和集群的负载均衡。同年,吴伟实现了一种基于 BP 神经网络的动态负载均衡算法
根据服务器的负载情况,预测请求的处理单元时间,并结合当先处理的请求数,进行其他的调度,两种方法其实都是预测的方法。

2020 秦娥基于服务器网站当前访问量的时间序列来预测未来高峰期的访问量\cite{qbee},对服务器集群中
各个节点在一天内总体剩余性能变化的趋势进行预测,从而调整服务器集群的节点进行调整,提升
了集群稳定性。

根据所学习的相关内容,并结合大学所进行的课程,使用人工智能分析访问量,并发量,并分析研究内置负载均衡算法;
通过比较使用两者后的测试结果,看是否能降低服务其的负载度和响应时间,进而给用户良好的网络体验,
具备在实际应用中的参考价值。

\section{论文结构}

本文详细结构如下:

第 1 章首先论述了本论文的研究背景和意义,国内外 Nginx 负载均衡算法研究现状,
以及自己研究的目的与意义。最后论述论文的大体结构。

第 2 章对所使用的相关技术进行介绍与研究,其中包括服务器集群技术,
负载均衡技术的类型和常见的负载调度策略,
并对Nginx重要模块和服务器集群中的反向代理技术做了介绍。

第 3 章 对人工智能算法的的分析与探究

第 4 章 搭建集群系统,利用相关工具对两种方法进行测试

第 5 章 分析结果

第 6 章 总结,参考文献,附录,致谢
