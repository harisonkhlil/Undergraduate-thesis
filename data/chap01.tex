%%================================================
%% Filename: chap01.tex
%% Encoding: UTF-8
%% Author: Su JunFeng
%% Email: harisonkhlil@gmail.com
%% Created: 2024-02-21
%% Last modified: 2024-02-22
%%================================================
\chapter{绪论}

\section{研究背景及意义}

\subsection{研究背景}

互联网的迅速发展和普遍应用极大地改变了人类生活的方方面面。互联网已渗透到日常生活的每个角落,从年轻一代的“数字土著”到逐渐接触网络的老年群体,大家纷纷融入这个数字时代,展示出显著的社交倾向、互动性及创新能力,共同塑造了多元且丰富的网络文化。
来自中国互联网络信息中心的第52次《中国互联网络发展状况统计报告》指出,截至2023年6月,中国网民人数已达到10.79亿,相比2021年12月增加了3549万,互联网覆盖率上升至75.6\%\cite{vsgohulmwhlofavjvlkltsjibcgc}

随着网络范围的持续扩展,服务于广大用户的服务器数量不断增加。人们利用互联网来接收信息、享受视频娱乐等活动越发频繁。
在此背景下,服务器后台处理能力面临重大考验。历史上,学者和工程师提出了多种方法来应对这一挑战,包括流量疏导、负载引流以及服务器集群等。
随着服务器集群规模的扩大,即服务器数量的增加,服务器间频繁的交互行为对处理能力提出了新的要求。
这些交互过程,特别是集群内部的数据交换和远程调用,成为了亟需解决的问题之一。此外,随着用户请求量的持续增加,如何保持各节点负载平衡也显得尤为重要。
为了维持节点负载的稳定性,采用负载均衡算法对任务进行智能分配,以实现资源的高效利用,已成为关键策略之一。

随着网络请求的不断攀升,提升服务器处理能力变得尤为关键。这一目标可以通过两个主要途径实现。首先是从硬件层面着手,通常称之为硬件级的负载均衡。主要手段是增添服务器节点,分散和处理更多的请求,以此提高整体的处理能力和效率。然而,这一策略可能伴随着高额的经济成本,对中小型企业来说是一笔沉重的财务压力,而且对于投入巨资购入的高性能服务器,如果未能充分发挥其性能,显然也会影响到成本效益比。
在许多情况下,企业可能会考虑一种更为经济高效的替代方案,即通过优化软件层面的算法来提升性能。这种优化不仅能节约资源并且有助于最大化现有服务器的性能潜力。通过智能分配和请求调度,可以在不增加硬件投入的同时,确保处理请求的效率和稳定性。\cite{qbee}

综上所述,对于非大型企业而言,从软件层面入手进行改进是一项切实可行的方案。
而对大型企业来说,软件层面的改进能够进一步提升庞大服务器集群的整体性能。
软件层面实现的负载均衡被称为软负载均衡,其原理是在一台或多台服务器的不同操作系统中安装一个或多个附加软件,
以达到负载均衡的目的。这种方法最显著的优势在于配置相对简单,且成本较为低廉。

对于软负载均衡常见的有 LVS\cite{lijp}, Nginx\cite{Zepeng}, HAProxy\cite{li2019dynamic}, Ribbon 等等。其中,Nginx 是一款优秀的软件负载均衡器,具有并发量高代码开源等优点,因此常常被用来作为服务器端的负载均衡器。
本文也是基于 Nginx 负载均衡系统的研究,在研究的基础上探究负载均衡算法并创造优化算法

\subsection{国内外研究现状}

在过去几十年中,互联网的蓬勃发展对Web服务器的并发处理能力提出了更高的要求,导致服务器面临的处理压力逐渐增大。
随着硬件负载均衡技术的发展,在软件负载均衡的领域,全球研究者也提出并实施了多项改进措施。在诸多服务器软件解决方案中,Nginx的出现尤其引人注目。

在国际领域,例如XIAONICHI等研究人员\cite{chi2012web},分深入探讨了负载均衡的各类技术。他们参考了历史上成功运用负载均衡技术的案例,聚焦于Nginx的关键功能模块以及其核心机制。通过将Nginx作为反向代理服务器,他们解决了并发请求处理的挑战,运用负载均衡和缓存技术策略有效应对服务器过载的问题。此外,研究者还根据具体需求定制网络拓扑结构,并将其应用于Discuz论坛系统。
通过实验和仿真分析,结果表明,利用Web缓存技术可以有效应对大量的并发请求,充分展现了其解决方案的可行性和功效。

多家国际企业持续深耕于集群系统负载调度器的研发。Cisco推出LocalDirect,引入最快响应算法与最小连接数算法。而IBM的Network Dispatcher则偏好将新请求分配至现有连接最少的服务器。Intel的网擎系统则采纳一种快速响应负载调度策略\cite{张淇2020服务器集群负载均衡算法在商务系统中的研究与应用}。

国内在这方面也不落人后,2016年,王永辉提出了新型的集群服务器分组管理方法。该方法规定服务器需定期收集并向集群控制节点上报负载指标信息。控制节点随后将这些数据反馈给负载均衡器。之后负载均衡器会计算出每台服务器的加权值,并从中选出权值最高的三台服务器。随后,基于权值与概率成正比的原则,负载均衡器从这三台服务器中随机选取一台来处理接入的客户端请求\cite{王永辉2015基于}。

在2018年,王东引入了一项创新的动态负载均衡算法,该算法基于UDP协议实现了服务器负载指标数据的多播传输。此算法以多核CPU一定时间内的平均工作负载为基础,来判定是否达到最大工作负荷阈值,并据此调整处于高负载状态下服务器的任务分配。
研究表明,这种方法确实提升了系统整体的处理能力,尽管如此,UDP的传输机制还是带来了某些可靠性方面的挑战\cite{王东2018动态反馈负载均衡策略的研究}。

紧接着在2021年,谭畅和胡磊等研究者展开了更进一步的工作\cite{谭畅2021云中心基于}。他们提出了一套动态权重调节的负载均衡策略,该策略在加权轮询算法的基础上进行了优化,兼顾了服务器的硬件性能以及实时负载状况。
经过一系列严格的测试,新策略在处理高并发请求时,无论是响应时间还是承担的实际并发量,均表现出超越原先算法的优异性能。

\subsection{研究目的与意义}

近年来,负载均衡研究领域涌现出众多创新思路,特别是在运用机器学习、深度学习以及神经网络等先进技术来预测并应对高并发场景,削减服务器压力方面。

2012年的一项研究采用多元回归分析,探究了系统响应时间与节点负载状态之间的实际关联,并据此构建了相关预测模型,以优化服务器负载分配。
与此同时,吴伟提出了一种基于BP神经网络的动态负载均衡算法,该算法能够根据服务器当前的负载数据预测处理请求所需的时间单元,并结合当前正在处理的请求数量进行智能调度。
这两种策略的本质都在于利用预测技术来实现其目标。

到了2020年,秦娥采用基于时间序列的分析方法,对服务器的访问量进行未来高峰期预测\cite{qbee}。此分析有助于服务器集群中各个节点在全天的性能余量趋势预测,进而实现集群中节点的合理配置,显著提升了整体的稳定性和效率。

结合所学知识及在大学期间的课程经验,运用智能算法来分析访问量和服务器负载指数,探究内置负载均衡算法的改进策略成为了研究的重点。通过这三方面的综合改进,可以有效减轻服务器负担,缩短响应时间,并为用户提供更加优质的网络体验。
这些研究成果不仅有理论意义,也具有实际应用中的价值。
\section{论文结构}

本文详细结构如下:

第一章开篇明确了本文的研究范围与深度,包括国内外Nginx负载均衡算法的发展现状,旨在明确研究动机及其学术与实践意义。篇末概述了整篇论文的结构安排。

在第二章中,详细阐述了研究过程中所依托的相关技术理论,涵盖了服务器集群技术、负载均衡的分类与核心调度策略。同时,对Nginx的关键模块以及服务器集群中的反向代理技术进行了深入分析。

第三章对时间序列分析及其算法执行进行了深入研究,并通过实验比较来筛选最合适的访问量预测算法。

第四章展开对负载均衡算法在神经网络领域的研讨,特别是如何将时间卷积网络算法与Nginx的负载均衡关键指标相结合,从而得到算法优化的方案及其具体实施步骤。

第五章提出了一种结合访问量预测和神经网络学习的新型动态加权轮询负载均衡算法,并探讨了此算法的理论基础及其具体实现方式。

第六章对整篇论文进行总结,审视了创新点及研究中存在的不足之处。
