%%================================================
%% Filename: abstract.tex
%% Encoding: UTF-8
%% Author: Yuan Xiaoshuai - yxshuai@gmail.com
%% Created: 2012-04-24 00:21
%% Last modified: 2019-11-01 09:26
%%================================================
\begin{cabstract}

	本研究围绕传统负载均衡算法的局限,即通常只能被动响应,无法主动预防由访问量激增引起的服务器性能下降,甚至是宕机,同时无法能够实时体现集群节点不能实时现出负载状态的现状。因此,本文提出了基于访问量和预测模型的Nginx服务器的负载均衡算法,主要贡献包括:

	(1)基于时间序列的网站访问量预测研究,用以提前预判访问量的变化趋势,并据此调整服务器节点,提供合适的服务器性能储备,以此来提高集群的稳定性和稳定性。本文利用Kaggle“Web Traffic Time Series Forcasting”比赛提供的数据集进行分析和预测,通过对比不同模型,确定了最适宜的预测模型,赋予负载均衡算法主动预测和调整的能力。。

	(2)预测服务器集群各节点的综合负载指标。选定能全面体现负载状况的指标,不断收集这些数据,并采用卷积神经网络构建预测模型,使用预测结果来指导负载均衡策略中的权重调整。

	(3)提出了基于真实负载情况的Nginx服务器负载均衡算法的改进策略,合理分配用户任务。创新性地使用了两阶段方法以解决神经网络初期数据缺乏的问题,并确保集群的稳定运行。同时,根据TCP拥塞控制的原理调整服务器节点权重,优化节点负载状况。

\end{cabstract}

\ckeywords{Nginx, 时间卷积网络, 负载均衡, 时间序列}

\begin{eabstract}

	This study revolves around the limitations of traditional load balancing algorithms, which typically can only react passively and are incapable of actively preventing server performance degradation or even crashes caused by surges in visitation. Furthermore, these algorithms fail to reflect the real-time load status of cluster nodes. Therefore, this paper proposes a load balancing algorithm for Nginx servers based on visitation volume and predictive models. The main contributions include:

	(1) A study on predicting website traffic volume based on time series, aiming to foresee the trend in traffic volume and accordingly adjust server nodes to provide adequate server performance reserves. This enhances the stability and resilience of the cluster. By analyzing and forecasting using the dataset provided by the Kaggle "Web Traffic Time Series Forecasting" competition and comparing different models, this paper identifies the most suitable prediction model, endowing the load balancing algorithm with the capability of proactive prediction and adjustment.

	(2) Predicting the comprehensive load indicators of server cluster nodes. By selecting indicators that fully reflect the load condition, continuously collecting these data, and employing convolutional neural networks to build prediction models, the results are used to guide the weight adjustments in load balancing strategies.

	(3) An innovative two-phase method is introduced to address the issue of initial data scarcity in neural networks, ensuring the stable operation of the cluster. Additionally, based on the principles of TCP congestion control, this paper presents an improved strategy for Nginx server load balancing algorithms, focusing on the equitable distribution of user tasks and optimizing node load conditions by adjusting server node weights.

\end{eabstract}

\ekeywords{Nginx, Time Convolutional Network, Load Balancing, Time Series}
