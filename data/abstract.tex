%%================================================
%% Filename: abstract.tex
%% Encoding: UTF-8
%% Author: Yuan Xiaoshuai - yxshuai@gmail.com
%% Created: 2012-04-24 00:21
%% Last modified: 2019-11-01 09:26
%%================================================
\begin{cabstract}

互联网的发展迅速,且携带的信息量巨大,信息交流速度迅速,自由程度高,并且实现了全球的信息共享。
但同时用户增加导致访问量暴增,很大成程度上增加了网络服务器的压力,同时也对服务器的性能提出了更高的要求。
由于大量的 Web 高并发访问对后台的服务器造成的压力问题亟待解决。
由此应运而生的服务器集群系统和负载均衡技术很大程度上缓解了这个问题,
并且这些技术一方面改善了服务器的性能,另一方面很大幅度地缩减了改善服务器性能所需要的开销。
现在服务器上应用的负载均衡技术有很多,有的分配策略也存在不足,可能会影响繁忙的服务器处理任务的同时还接收着大量访问请求,
而部分服务器却处于空闲状态未分配到请求任务,这种现象会影响整个服务器集群对高并发访问的响应速度,使得用户体验变差。

针对服务器集群中负载均衡策略存在的分配不合理问题,本文研究了基于 Nginx 高并发服务器的负载均衡算法,其主要工作如下:

(1)基于服务器网站访问量的时间序列预测研究。主要工作,对传统负载均衡算法只能被动,不能主动进行防御的缺陷,其访问量激增导致的服务器集群性能下降甚至宕机的问题,通过使用数学分析和机器学习算法对网站未来访问量进行预测,根据预测结果对服务器节点进行调整,提升集群的稳定性。并利用 Kaggle 比赛“Web Traffic Time Series Forcasting”所提供的网站访问量数据集,对数据进行分析和预测,并通过两个模型的对比,确定了合适的预测模型,使得负载均衡算法具有主动探测的能力。

(2)对服务器集群各个节点在一天内的总体综合负载指标变化趋势进行预测。选取合适的指标作为能够体现综合负载情况的依据,并在各个节点不断收集负载信息数据,将得到的各个节点的负载数据进行分析和生成时间序列样本数据,使用卷积神经网络建立综合负载预测模型。并将其预测结果作为后面负载均衡策略中权重调整的依据。

(3)基于 Nginx 高并发访问的负载均衡的改进。针对Nginx服务器传统的负载均衡无法真正得到不同服务器节点的真实负载情况的问题,提出了一种改进的动态负载均衡算法对用户的任务进行合理的权重分配。创新的使用两个阶段来解决神经网络初期没有数据的问题,同时保证了集群的稳定运行。另一个阶段采用了对节点的综合负载指标预测结果来判断是否调整某个服务器当前的权重。参考了TCP拥塞控制的原理,来对各个服务器节点的权重进行调整,从而达到控制节点负载状态的目的。

\end{cabstract}

\ckeywords{Nginx, 时间卷积网络, 负载均衡,时间序列预测算法}

\begin{eabstract} 
The rapid development of the internet carries a massive volume of information, enables swift communications, fosters a high degree of freedom, and has achieved global information sharing. However, the increasing number of users has led to a surge in access demand, significantly burdening network servers and raising the performance standards required of them. The pressure on backend servers due to the high concurrency of web access urgently needs to be addressed. Server cluster systems and load balancing technology have emerged in response to this issue, greatly mitigating it. These technologies not only enhance server performance but also substantially reduce the costs associated with server performance improvements. Many load balancing technologies are currently applied to servers, yet some distribution strategies are insufficient, potentially affecting busy servers that continue to receive an overload of access requests while other servers remain idle with no assigned tasks. This disparity can affect the entire server cluster's response time to high-concurrency access, leading to a degraded user experience.

In response to the issue of improper distribution in server cluster load balancing strategies, this paper investigates a load balancing algorithm based on the high-concurrency server Nginx, with the following main contributions:

(1) Research on time-series forecasting of website traffic volume. This addresses the traditional load balancing algorithms' passive nature by employing mathematical analysis and machine learning algorithms to predict future website traffic. Adjustments to server nodes based on these predictions enhance cluster stability. Using the dataset provided by the Kaggle competition "Web Traffic Time Series Forecasting," data analysis and prediction are conducted, and through the comparison of two models, an appropriate forecasting model is selected, imbuing the load balancing algorithm with proactive detection capabilities.

(2) Prediction of the overall composite load index trends for each node in a server cluster throughout a day. Suitable indicators reflecting the composite load are selected, and load data are continuously collected from each node. This data is analyzed and used to generate time-series sample data, and a convolutional neural network is employed to establish a comprehensive load prediction model. The results of this prediction serve as the basis for weight adjustments in subsequent load balancing strategies.

(3) Improvements to load balancing under high-concurrency access based on Nginx. Addressing the issue of traditional Nginx server load balancing algorithms failing to accurately ascertain the real load conditions of different server nodes, a novel dynamic load balancing algorithm proposes a rational weight distribution of user tasks. An innovative two-phase approach overcomes the lack of initial data for neural networks while ensuring cluster stability. One phase involves adjusting the weight of a server based on predicted comprehensive load index results for that node. Drawing on the principles of TCP congestion control, adjustments to the weight of each server node are made to manage the node's load condition effectively.

\end{eabstract}

\ekeywords{Nginx, Time Convolutional Network, Load Balancing, Time Series Forecasting Algorithm}
