%%================================================
%% Filename: chap02.tex
%% Encoding: UTF-8
%% Author: Yuan Xiaoshuai - yxshuai@gmail.com
%% Created: 2012-04-27 19:37
%% Last modified: 2019-11-05 23:57
%%================================================
\chapter{模板简介及安装}
\label{cha:introduction}

\zzuthesis(\textbf{Z}heng\textbf{z}hou \textbf{U}niversity \textbf{Thesis})以
ctexbook文档类为基础,根据清华大学学位论文模板修改而来。本科毕业设计(论文)
和研究生学位论文(含硕士和博士)分别根据《郑州大学材料科学与工程学院本科毕业
设计(论文)基本规范》和《郑州大学学位论文写作规范格式》的规范要求定制,目前
该模板已基本符合相关《规范》的要求。

\textsf{模板作者声明:}该模板非学校官方模板,不能保证满足《规范》所有要求,遇到问题请及时反馈。
  
\section{模板简介}
\label{sec:intro}

下表列出了 \zzuthesis{} 的主要文件及其功能介绍。模板核心文件只有两个:\texttt{zzuthesis.cls} 和 \texttt{zzubib.bst}。其中 \texttt{zzuthesis.cls} 文件定义了模版所需要的命令,\texttt{zzubib.bst} 文件则定义了参考文献的风格。由于学校规范中对参考文献风格的规定较为模糊,建议直接采用最新的国家标准《信息与文献参考文献著录规则》(\emph{GB/T 7714-2015}) 中规定的参考文献格式,对该问题的讨论见 https://github.com/tuxify/zzuthesis/issues/2。

\begin{longtable}{lp{8cm}}
\toprule
{\heiti 文件(夹)} & {\heiti 功能描述}\\\midrule
\endfirsthead
\midrule
{\heiti 文件(夹)} & {\heiti 功能描述}\\\midrule
\endhead
\endfoot
\endlastfoot
zzuthesis.cls & 模板类文件\\
zzubib.bst & 参考文献样式文件\\
docutils.sty & 模板示例文档用到的宏包及定义\\ 
main.tex & 示例文档主文件\\
spine.tex & 书脊示例文档\\
a3cover.tex & A3封面示例文档\\
ref/ & 示例文档参考文献目录\\
data/ & 示例文档章节具体内容\\
figures/ & 示例文档图片路径\\
Makefile & Linux 自动编译工具\\
\bottomrule
\end{longtable}

\section{模板下载及安装}

该模板可以通过浏览器下载\textbf{发布版本},或者通过 git 命令下载\textbf{开发版本}。下载后进入模板主目录即可编译生成示例文档,模板在 \TeXLive{} 2019 下编译通过,建议采用最新版本的 \TeXLive{} 软件发行套装\footnote{软件的安装参见:\url{http://tug.org/texlive/acquire.html}}。
\begin{itemize}
\item 浏览器下载:
\href{https://github.com/tuxify/zzuthesis/releases}{Releases}
\item git 命令:
\texttt{git clone https://github.com/tuxify/zzuthesis.git}
\end{itemize}

\section{模板编译运行}

模板中示例文档的生成可以采用多种方法,手动运行需要多次运行命令,直到不再出现警告为止,具体步骤如下:
\begin{shell}
# 1. 发现引用关系,文件后缀 .tex 可以省略
$ xelatex main
# 2. 编译参考文件源文件,生成 bbl 文件
$ bibtex main
# 3. 解决引用
$ xelatex main
$ xelatex main   # 此时生成完整的 PDF 文件
\end{shell}

采用 latexmk 命令可以支持全自动生成文档,会自动运行多次工具直到交叉引用都被解决。
下面给出了一个用 latexmk 调用 \XeLaTeX{} 生成最终文档的示例:
\begin{shell}
# 一句命令即可生成完整 PDF 文件
$ latexmk -xelatex main
\end{shell}

在Linux平台还可以使用 \emph{Makefile} 文件,在本模板中,该文件实质上也是调用了 latexmk 命令。
\begin{shell}
$ make           # 生成示例文档、书脊及 A3 封面
$ make thesis    # 生成示例文档
$ make a3cover   # 生成 A3 封面
$ make clean     # 清除临时文件
\end{shell}

此外,习惯用 Visual Studio Code 编辑器的还可以借助 LaTeX Workshop 插件,实现自动化编译运行,具体设置可参考网上相关教程。